\input{doc-class-cours.tex}

\begin{document}

Learn the basics of Go, an open source programming language originally developed by a team at Google and enhanced by many contributors from the open source community. This course is designed for individuals with previous programming experience using such languages as C, Python, or Java, and covers the fundamental elements of Go. Topics include data types, protocols, formats, and writing code that incorporates RFCs and JSON. Most importantly, you’ll have a chance to practice writing Go programs and receive feedback from your peers. Upon completing this course, you'll be able to implement simple Go programs, which will prepare you for subsequent study at a more advanced level.

\section*{Semaine 1}

\subsection*{Introduction to the Specialization}

\begin{itemize}[label={$\bullet$}]
  \item Reading: Specialization Overview
\end{itemize}    

\subsection*{Introduction to the Course}

\begin{itemize}[label={$\bullet$}]
  \item Vidéo: Welcome to the Course
  \item Reading: Go Documentation
\end{itemize}

\subsection*{Module 1: Getting Started with Go}

This first module gets you started with Go. You'll learn about the advantages of using Go and begin exploring the language's features. Midway through the module, you’ll take a break from "theory" and install the Go programming environment on your computer. At the end of the module, you'll write a simple program that displays “Hello, World” on your screen.
\begin{multicols}{2}
\begin{itemize}[label={$\bullet$}]
  \item Vidéo: Module 1 Overview
  \item Vidéo: M1.1.1 - Why Should I Learn Go ? (Advantages of Go)
  \item Vidéo: M1.1.2 - Objects
  \item Vidéo: M1.1.3 - Concurrency
  \item Vidéo: M1.2.1 - Installing Go
  \item Vidéo: M1.2.2 - Workspaces \& Packages
  \item Vidéo: M1.2.3 - Go Tool
  \item Vidéo: M1.3.1 - Variables
  \item Vidéo: M1.3.2 - Variable Initialization
  \item Noté: Module 1 Activity: "Hello, world!"
  \item Noté: Module 1 Quiz
\end{itemize}    
\end{multicols}    
\section*{Semaine 2}

\subsection*{Module 2: Basic Data Types}

Now that you’ve set up your programming environment and written a test program, you’re ready to dive into data types. This module introduces data types in Go and gives you practice writing routines that manipulate different kinds of data objects, including floating-point numbers and strings.
\begin{multicols}{2}
\begin{itemize}[label={$\bullet$}]
  \item Reading: STOP -Read This First!
  \item Vidéo: Module 2 Overview
  \item Vidéo: M2.1.1 - Pointers
  \item Vidéo: M2.1.2 - Variable Scope
  \item Vidéo: M2.1.3 - Deallocating Memory
  \item Vidéo: M2.1.4 - Garbage Collection 
  \item Vidéo: M2.2.1 - Comments, Printing, Integers
  \item Vidéo: M2.2.2 - Ints, Floats, Strings
  \item Vidéo: M2.2.3 - String Packages
  \item Vidéo: M2.3.1 - Constants
  \item Vidéo: M2.3.2 - Control Flow
  \item Vidéo: M2.3.3 - Control Flow, Scan
  \item Noté: Module 2 Activity: trunc.go
  \item Noté: Module 2 Activity: findian.go
  \item Noté: Module 2 Quiz
\end{itemize}
\end{multicols}
\section*{Semaine 3}

\subsection*{Module 3: Composite Data Types}

At this point, we’re ready to move into more complex data types, including arrays, slices, maps, and structs. As in the previous module, you’ll have a chance to practice writing code that makes use of these data types.
\begin{multicols}{2}
\begin{itemize}[label={$\bullet$}]
  \item Vidéo: Module 3 Overview
  \item Vidéo: M3.1.1 - Arrays
  \item Vidéo: M3.1.2 - Slices
  \item Vidéo: M3.1.3 - Variable Slices
  \item Vidéo: M3.2.1 - Hash Tables
  \item Vidéo: M3.2.2 - Maps
  \item Vidéo: M3.3.1 - Structs
  \item Noté: Module 3 Activity: slice.go
  \item Noté: Module 3 Quiz
\end{itemize}
\end{multicols}
\section*{Semaine 4}

\subsection*{Module 4: Protocols and Formats}

This final module of the course introduces the use of remote function calls (RFCs) and JavaScript Object Notation (JSON) in Go. You’ll learn how to access and manipulate data from external files, and have an opportunity to write several routines using Go that exercise this functionality.
\begin{multicols}{2}
\begin{itemize}[label={$\bullet$}]
  \item  Vidéo: Module 4 Overview
  \item  Vidéo: M4.1.1 - RFCs
  \item  Vidéo: M4.1.2 - JSON
  \item  Vidéo: M4.2.1 - File Access, ioutil
  \item  Vidéo: M4.2.2 - File Access, os
  \item Noté: Module 4 Activity: makejson.go
  \item Noté: Final Course Activity: read.go
\end{itemize}
\end{multicols}

\end{document}