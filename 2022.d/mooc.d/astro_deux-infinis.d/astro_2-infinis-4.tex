\documentclass[11pt]{article}
\usepackage{geometry,marginnote} % Pour passer au format A4
\geometry{hmargin=1cm, vmargin=1cm} % 

% Page et encodage
\usepackage[T1]{fontenc} % Use 8-bit encoding that has 256 glyphs
\usepackage[english,french]{babel} % Français et anglais
\usepackage[utf8]{inputenc} 

\usepackage{lmodern,numprint}
\setlength\parindent{0pt}

% Graphiques
\usepackage{graphicx,float,grffile}
\usepackage{pst-eucl, pst-plot,units} 

% Maths et divers
\usepackage{amsmath,amsfonts,amssymb,amsthm,verbatim}
\usepackage{multicol,enumitem,url,eurosym,gensymb}
\DeclareUnicodeCharacter{20AC}{\euro}

% Sections
\usepackage{sectsty} % Allows customizing section commands
\allsectionsfont{\centering \normalfont\scshape}

% Tête et pied de page
\usepackage{fancyhdr} \pagestyle{fancyplain} \fancyhead{} \fancyfoot{}

\renewcommand{\headrulewidth}{0pt} % Remove header underlines
\renewcommand{\footrulewidth}{0pt} % Remove footer underlines

\newcommand{\horrule}[1]{\rule{\linewidth}{#1}} % Create horizontal rule command with 1 argument of height

\newcommand{\Pointilles}[1][3]{%
  \multido{}{#1}{\makebox[\linewidth]{\dotfill}\\[\parskip]
}}

\newtheorem{Definition}{Définition}

\usepackage{siunitx}
\sisetup{
    detect-all,
    output-decimal-marker={,},
    group-minimum-digits = 3,
    group-separator={~},
    number-unit-separator={~},
    inter-unit-product={~}
}

\setlength{\columnseprule}{1pt}

\begin{document}

Partez à la découverte de l'infiniment grand et de l'infiniment petit dans leurs aspects les plus proches de notre quotidien, en compagnie de physiciens et de physiciennes qui vont vous faire découvrir leur présence insoupçonnée dans notre vie de tous les jours. Vous vous initierez à la vie et aux métiers d'une grande collaboration en physique de l'infiniment petit et de l'infiniment grand, vous découvrirez comment les outils développés dans ces domaines ont trouvé des applications inattendues, comment la physique nucléaire a profondément modifié les domaines de l'énergie et de la santé, et comment les propriétés de certaines particules aident à présent d'autres disciplines à sonder la matière d'une manière totalement différente.

\section*{Semaine 1 - Des Infinis et des Hommes}

\textit{Ce module s'intéresse à la nature de la recherche fondamentale et à ses liens avec notre quotidien. Il aborde aussi la manière dont travaillent les hommes et les femmes qui participent aux progrès de la connaissance.}


\begin{itemize}[label={$\bullet$}]
    \item Vidéo : Deux infinis proches de nous
    \item Vidéo : Les grandes collaborations
    \item Vidéo : Les métiers de la recherche
  \end{itemize}

\section*{Semaine 2 - Des technologies dans notre quotidien}

\textit{Ce module décrit des technologies qui sont présentes dans notre quotidien et qui présentent des liens forts avec la physique de l'infiniment grand et de l'infiniment petit, tantôt par le biais des concepts, tantôt par celui des outils.}
  
  
\begin{itemize}[label={$\bullet$}]
    \item Vidéo : Le GPS
    \item Vidéo : Les semi-conducteurs, du transistor aux détecteurs
    \item Vidéo : Du web au big data
\end{itemize}

\section*{Semaine 3 - De nouveaux regards sur le monde}

\textit{Ce module montre comment les propriétés de certaines particules ont pu être exploitées pour mieux sonder la structure de la matière, en utilisant le rayonnement synchrotron, les neutrinos ou les muons.}
    
    
\begin{itemize}[label={$\bullet$}]
    \item Vidéo : Neutrinos et muons, des particules pleines de ressources
    \item Vidéo : Le rayonnement synchrotron
\end{itemize}    

\section*{Semaine 4 - Les noyaux au service de l'énergie et de la santé}

\textit{Ce module décrit des applications importantes de la physique nucléaire afin de produire de l'énergie ou bien d'aider la recherche médicale par de nouvelles techniques d'imagerie et de soin.}
      
      
\begin{itemize}[label={$\bullet$}]
    \item Vidéo : L'énergie nucléaire
    \item Vidéo : La médecine et les deux infinis
\end{itemize}

\end{document}