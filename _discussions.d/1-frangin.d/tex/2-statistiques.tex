\documentclass[11pt]{article}
\usepackage{geometry} % Pour passer au format A4
\geometry{hmargin=1cm, vmargin=1cm} % 

% Page et encodage
\usepackage[T1]{fontenc} % Use 8-bit encoding that has 256 glyphs
\usepackage[english,francais]{babel} % Français et anglais
\usepackage[utf8]{inputenc} 

\usepackage{lmodern}
\setlength\parindent{0pt}

% Graphiques
\usepackage{graphicx, float}
\usepackage{tikz,tkz-tab}

% Maths et divers
\usepackage{amsmath,amsfonts,amssymb,amsthm,verbatim}
\usepackage{multicol,enumitem,url,eurosym,gensymb}

% Sections
\usepackage{sectsty} % Allows customizing section commands
\allsectionsfont{\centering \normalfont\scshape}

% Tête et pied de page

\usepackage{fancyhdr} 
\pagestyle{fancyplain} 

\fancyhead{} % No page header
\fancyfoot{}

\renewcommand{\headrulewidth}{0pt} % Remove header underlines
\renewcommand{\footrulewidth}{0pt} % Remove footer underlines

\newcommand{\horrule}[1]{\rule{\linewidth}{#1}} % Create horizontal rule command with 1 argument of height

%----------------------------------------------------------------------------------------
%	Début du document
%----------------------------------------------------------------------------------------

\begin{document}

%----------------------------------------------------------------------------------------
% RE-DEFINITION
%----------------------------------------------------------------------------------------
% MATHS
%-----------

\newtheorem{Definition}{Définition}
\newtheorem{Theorem}{Théorème}
\newtheorem{Proposition}{Propriété}

% MATHS
%-----------
\renewcommand{\labelitemi}{$\bullet$}
\renewcommand{\labelitemii}{$\circ$}
%----------------------------------------------------------------------------------------
%	Titre
%----------------------------------------------------------------------------------------

\setlength{\columnseprule}{1pt}

\section{Statistique}

Dans le doute, je pars du début.

\subsection{Exercice 6}

\begin{enumerate}
  \item[1.] On étudie une population de 50 souris. (Je pense que le nombre est important. On étudie pas seulement la / des souris.) La taille de l'échantillon est 35. La variable statistique étudié est le nombre de souris noires parmi 50 desecendants.

  \item[2a.]
  \begin{center}
    \begin{tabular}{|c||c|c|c|c|c|c|c|c|c|c|c|c|c||c|}
      \hline
      $x_i$ & 18 & 19 & 20 & 21 & 22 & 23 & 24 & 25 & 26 & 27 & 28 & 29 & 32 & Total  \\
      \hline
      $n_i$ &  1 &  1 &  1 &  2 &  4 &  4 &  5 &  9 &  2 &  2 &  1 &  2 &  1 & 35 \\
      \hline
      $n_c$ &  1 &  2 &  3 &  5 &  9 & 13 & 18 & 27 & 29 & 31 & 32 & 34 &  35 & 35 \\
      \hline      
    \end{tabular}
  \end{center}

  \item[2b.] Le mode de la série est 25. \\
  $\max - \min = 32 - 18 = 14$.\\
  L'étendue de la série est 14.

  \item[2c.] La médiane sépare la série ordonnée en 2 parties de même effectifs. L'effectif total de la série est 35. La médiane est done la 18ième valeur car $35 = 17 + 1 + 17$.\\

  Dans le tableau la 18ième valeur est 24 descendants de couleur noire.\\

  Les quartiles séparent la série ordonnée à au moins $\frac{1}{4}$ de la série est plus petite puis à $\frac{3}{4}$.

  Q1 : $35 \times \dfrac{1}{4} = 8.75$ (de tête , ce calcul est loin d'être drôle par contre, c'est assez simple de voir que c'est plus petit que 9 car $4\times9 = 36$.) \\

  On prend la 9ième valeur : 22.\\
  Q1 est 22 descendants de couleur noire.\\

  Q3 : $35 \times \dfrac{3}{4} = 26.25$ (de tête , ce calcul est loin d'être drôle par contre, Mais on peut savoir que 3 c'est pas très loin (toujours en plus petit) que 3 fois le précédent)\\

  On prend la 27ième valeur : 25.\\
  Q3 est 25 descendants de couleur noire.\\

  \item[3a.] (Je suis assez sceptique sur ce genre de calcul à faire sans calculatrice...)

  \begin{center}
    \begin{tabular}{|c||c|c|c|c|c|c|c||c|}
      \hline
      $x_i$ & [18, 20[ & [20, 22[ & [22, 24[ & [24, 26[ & [26, 28[& [28, 30[& [30, 32[ & Total \\
      \hline
      $n_i$ &        2 &        3 &        8 &      14  &       4 &       3 &       1  & 35  \\
      \hline
      $n_c$ &        2 &        5 &       13 &      27  &      31 &      34 &      35  & 35 \\ 
      \hline      
      $c_i$ &       19 &       21 &       23 &      25  &      27 &      29 &      31  &  \\ 
      \hline  
      $n_i c_i$ &   38 &       63 &      184 &     350  &     108 &      87 &      31  & 861 \\ 
      \hline  
      $n_i c_i^2$ & 722&     1323 &     4232 &     8750 &    2916 &    2523 &      961 & 21429 \\
      \hline   
    \end{tabular}
  \end{center}

  \item[3b.] Je fais les graphiques en scan, ça ira plus vite pour moi. Attention avec les histogrammes ; il y a un notion de proportionnalité d'aire du rectangle... En gros, il faut faire gaffe quand les classes n'ont pas la même largeur... Sinon, on s'en fout plutôt.
  Pour les gens tatillons, il faut que l'aire du rectangle soit égale à ton $n_i$...
  En vrai, on n'en fait rarement et on préfère les diagrammes en batons (même pour des classes...)

  La classe modale est $[24, 26[$.

  \item[3c.] 
  Moyenne
  \begin{eqnarray*}
    m &=& \sum_i \dfrac{n_i c_i}{total} \text{ // Pas certain qu'ils attendent cette notation.} \\
    m &=& \dfrac{n_1 c_1 + ... + n_7 c_7 }{total} \\
    m &=& \dfrac{38 + ... + 31}{35} \\
    m &=& \dfrac{861}{35} \\
    m &=& 24.6
  \end{eqnarray*}

  Avec des classes de largeurs deux, en moyenne il y a 24.6 descendants noirs.

ecart-type

\begin{eqnarray*}
s &=& \sqrt { \frac {1}{total} \sum _i (c_i - m)^2} \\
s &=& \sqrt { \frac {1}{total} ( \sum _i (n_i c_i^2))  - m^2} \\
s & \approx & \sqrt { \frac {1}{35} \times (722 + ... + 961) - 24.6^2 } \\
s & \approx & \sqrt { \frac {1}{35} \times (21427) - 24.6^2 } \\
s & \approx & \sqrt { 612.2 - 24.6^2 } \\
s & \approx & \sqrt {7.04} \\
s & \approx & 2.65 \\
\end{eqnarray*}

L'écart-type est de 2.65

%data = c(19, 19, 21, 21, 21, rep(23,8), rep(25,14), 27,27,27,27,29,29,29,31)



\end{enumerate}
\end{document}
