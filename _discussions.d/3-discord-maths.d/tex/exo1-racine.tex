\documentclass[11pt]{article}
\usepackage{geometry} % Pour passer au format A4
\geometry{hmargin=1cm, vmargin=1cm} % 

% Page et encodage
\usepackage[T1]{fontenc} % Use 8-bit encoding that has 256 glyphs
\usepackage[english,francais]{babel} % Français et anglais
\usepackage[utf8]{inputenc} 

\usepackage{lmodern}
\setlength\parindent{0pt}

% Graphiques
\usepackage{graphicx, float}
\usepackage{tikz,tkz-tab}

% Maths et divers
\usepackage{amsmath,amsfonts,amssymb,amsthm,verbatim}
\usepackage{multicol,enumitem,url,eurosym,gensymb}

% Sections
\usepackage{sectsty} % Allows customizing section commands
\allsectionsfont{\centering \normalfont\scshape}

% Tête et pied de page

\usepackage{fancyhdr} 
\pagestyle{fancyplain} 

\fancyhead{} % No page header
\fancyfoot{}

\renewcommand{\headrulewidth}{0pt} % Remove header underlines
\renewcommand{\footrulewidth}{0pt} % Remove footer underlines

\newcommand{\horrule}[1]{\rule{\linewidth}{#1}} % Create horizontal rule command with 1 argument of height

%----------------------------------------------------------------------------------------
% Début du document
%----------------------------------------------------------------------------------------

\begin{document}

$$\sqrt{1 - x } + \sqrt{1 + x } + 2\sqrt{1 - x^2 } = 4$$

\setlength{\columnseprule}{1pt}
\begin{multicols}{2}

\subsection*{domaine de définition}

\textit{des racines donc des problèmes pour les valeurs négatives.}

\begin{eqnarray*}
1 - x &\geq& 0 \\
1 + x &\geq& 0 \\
1 - x^2 &\geq& 0
\end{eqnarray*}

On cherche $x \in [-1 ; 1]$

\subsection*{Factorisation}

On sait que $a^2 + b^2 + 2ab = (a+b)^2$ \\

On prend $a = \sqrt{1 - x }$ et $b = \sqrt{1 + x }$.\\

De plus : 

\begin{eqnarray*}
ab &=& \sqrt{1 + x } \times \sqrt{1 - x } \\
ab &=& \sqrt{(1+x)(1-x)} \\
ab &=& \sqrt{1 - x^2}
\end{eqnarray*}

On factorise notre équation.

\begin{eqnarray*}
\sqrt{1 - x } + \sqrt{1 + x } + 2\sqrt{1 - x^2 } &=& 4 \\
(\sqrt{1-x} + \sqrt{1+x})^2 &=& 4 \\
 \left\{
      \begin{aligned}
\sqrt{1-x} + \sqrt{1+x} &=& 2 \\
\sqrt{1-x} + \sqrt{1+x} &=& -2 \\
      \end{aligned}
    \right.
\end{eqnarray*}

\textit{ // impossible, une somme de racine de ne peut être négative dans R.}

\subsection*{Changement de variable}

$$\sqrt{1-x} + \sqrt{1+x} = 2$$

On pose  $y = \sqrt{1-x}$

On a alors : 

\begin{eqnarray*}
y &=& \sqrt{1-x} \\
y^2 &=& 1-x \\
x &=& 1-y^2 \\ 
\end{eqnarray*}

On obtient aussi :

\begin{eqnarray*} 
\sqrt{1+x} &=& \sqrt{1+1-y^2} \\
\sqrt{1+x} &=& \sqrt{2-y^2} 
\end{eqnarray*}

d'où notre équation : 

\begin{eqnarray*} 
\sqrt{1-x} + \sqrt{1+x}) &=& 2 \\
y + \sqrt{2-y^2}  &=& 2  \\
y - 2 &=& - \sqrt{2-y^2}\\
(y - 2)^2 &=& 2 - y^2\\
y^2 -4y + 4 &=& 2 - y^2\\
2y^2 - 4y + 2 &=& 0
\end{eqnarray*}

\subsection*{Polynôme du second degré}

Calcul du discriminant

\begin{eqnarray*} 
\Delta &=& 4^4 - 4\times 2\times 2 \\
\Delta &=& 0
\end{eqnarray*}

Une seule racine double.
\begin{eqnarray*} 
y &=& -\dfrac{-4}{2\times2} \\
y &=& 1
\end{eqnarray*}

\subsection*{Solution}

\begin{eqnarray*} 
y &=& 1 \\\
\sqrt{1-x} &=& 1 \\ 
1-x &=& 1\\
x &=& 0
\end{eqnarray*}

x=0 est bien l'unique putain de solution qu'on avait déjà trouver visuellement... mais au moins, on sait qu'il y en a pas d'autre... Même si le tracer graphique de la fonction nous donnait bien le résultat aussi...

\end{multicols}

\end{document}  
