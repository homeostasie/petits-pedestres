\documentclass[12pt]{article}
\usepackage{geometry} \geometry{hmargin=1cm, vmargin=1cm} % 

% Page et encodage
\usepackage[T1]{fontenc} % Use 8-bit encoding that has 256 glyphs
\usepackage[english,french]{babel} % Français et anglais
\usepackage[utf8]{inputenc} 

\usepackage{lmodern}
\setlength\parindent{0pt}

% Graphiques
\usepackage{graphicx,float,grffile}

% Maths et divers
\usepackage{amsmath,amsfonts,amssymb,amsthm,verbatim}
\usepackage{multicol,enumitem,url,eurosym,gensymb}

% Sections
\usepackage{sectsty} \allsectionsfont{\centering \normalfont\scshape}

% Tête et pied de page

\usepackage{fancyhdr} \pagestyle{fancyplain} \fancyhead{} \fancyfoot{}
\renewcommand{\headrulewidth}{0pt} \renewcommand{\footrulewidth}{0pt} 

\usepackage{hyperref}

\begin{document}

\newtheorem{Theorem}{Théorème}

\subsection*{1 - Vocabulaire}
\begin{Theorem}{Expérience aléatoire} \label{thm1-ea}\\
Une expérience est aléatoire si elle vérifie deux conditions.


\begin{itemize}
\item On connait toutes les \textbf{issues} possibles. \textit{On sait ce qui peut se passer.}
\item Le résultat n'est pas \textbf{prévisible}. \textit{On ne sait pas ce qui va se passer.}
\end{itemize}

\end{Theorem}

\newpage
blabla
\newpage

J'ai envie de me rappeler de ça : \hyperref[thm1-ea]{Théorème blabla}



\end{document}

