\documentclass[11pt]{article}
\usepackage{geometry,marginnote} % Pour passer au format A4
\geometry{hmargin=1cm, vmargin=1cm} % 

% Page et encodage
\usepackage[T1]{fontenc} % Use 8-bit encoding that has 256 glyphs
\usepackage[english,french]{babel} % Français et anglais
\usepackage[utf8]{inputenc} 

\usepackage{lmodern,numprint}
\setlength\parindent{0pt}

% Graphiques
\usepackage{graphicx,float,grffile}
\usepackage{pst-eucl, pst-plot,units} 

% Maths et divers
\usepackage{amsmath,amsfonts,amssymb,amsthm,verbatim}
\usepackage{multicol,enumitem,url,eurosym,gensymb}
\DeclareUnicodeCharacter{20AC}{\euro}

% Sections
\usepackage{sectsty} % Allows customizing section commands
\allsectionsfont{\centering \normalfont\scshape}

% Tête et pied de page
\usepackage{fancyhdr} \pagestyle{fancyplain} \fancyhead{} \fancyfoot{}

\renewcommand{\headrulewidth}{0pt} % Remove header underlines
\renewcommand{\footrulewidth}{0pt} % Remove footer underlines

\newcommand{\horrule}[1]{\rule{\linewidth}{#1}} % Create horizontal rule command with 1 argument of height

\newcommand{\Pointilles}[1][3]{%
  \multido{}{#1}{\makebox[\linewidth]{\dotfill}\\[\parskip]
}}

\newtheorem{Definition}{Définition}

\usepackage{siunitx}
\sisetup{
    detect-all,
    output-decimal-marker={,},
    group-minimum-digits = 3,
    group-separator={~},
    number-unit-separator={~},
    inter-unit-product={~}
}

\setlength{\columnseprule}{1pt}

\begin{document}


Learn the basics of Go, an open source programming language originally developed by a team at Google and enhanced by many contributors from the open source community. This course is designed for individuals with previous programming experience using such languages as C, Python, or Java, and covers the fundamental elements of Go. Topics include data types, protocols, formats, and writing code that incorporates RFCs and JSON. Most importantly, you’ll have a chance to practice writing Go programs and receive feedback from your peers. Upon completing this course, you'll be able to implement simple Go programs, which will prepare you for subsequent study at a more advanced level.

\section*{Semaine 1 - Introduction}

\subsection*{Introduction to the Course}

Learn the basics of Go, an open source programming language originally developed by a team at Google and enhanced by many contributors from the open source community. This is the first in a series of three courses comprising the Programming with Google Go specialization. It is designed for individuals with previous programming experience using such languages as C, Python, or Java, and covers the fundamental elements of Go. Topics include data types, protocols, formats, and writing code that incorporates RFCs and JSON. Most importantly, you’ll have a chance to practice writing Go programs and receive feedback from your peers. Upon completing this course, you’ll be able to implement simple Go programs, which will prepare you for the remaining two courses in this specialization: Functions, Methods, and Interfaces in Go and Concurrency in Go.

\begin{itemize}[label={$\bullet$}]
    \item Welcome to the Course
    \item Reading: Go Documentation
\end{itemize}

\subsection*{Module 1: Getting Started with Go}

This first module gets you started with Go. You'll learn about the advantages of using Go and begin exploring the language's features. Midway through the module, you’ll take a break from "theory" and install the Go programming environment on your computer. At the end of the module, you'll write a simple program that displays “Hello, World” on your screen.

\begin{itemize}[label={$\bullet$}]
   \item  Module 1 Overview
   \item  M1.1.1 - Why Should I Learn Go? (Advantages of Go)
   \item  M1.1.2 - Objects
   \item  M1.1.3 - Concurrency
   \item  M1.2.1 - Installing Go
   \item  M1.2.2 - Workspaces \& Packages
   \item  M1.2.3 - Go Tool
   \item  M1.3.1 - Variables
   \item  M1.3.2 - Variable Initialization
   \item Noté: Module 1 Activity: "Hello, world!"
   \item Noté: Module 1 Quiz
\end{itemize}


\end{document}