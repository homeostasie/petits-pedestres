\documentclass[11pt]{article}
\usepackage{geometry,marginnote} % Pour passer au format A4
\geometry{hmargin=1cm, vmargin=1cm} % 

% Page et encodage
\usepackage[T1]{fontenc} % Use 8-bit encoding that has 256 glyphs
\usepackage[english,french]{babel} % Français et anglais
\usepackage[utf8]{inputenc} 

\usepackage{lmodern,numprint}
\setlength\parindent{0pt}

% Graphiques
\usepackage{graphicx,float,grffile}
\usepackage{pst-eucl, pst-plot,units} 

% Maths et divers
\usepackage{amsmath,amsfonts,amssymb,amsthm,verbatim}
\usepackage{multicol,enumitem,url,eurosym,gensymb}
\DeclareUnicodeCharacter{20AC}{\euro}

% Sections
\usepackage{sectsty} % Allows customizing section commands
\allsectionsfont{\centering \normalfont\scshape}

% Tête et pied de page
\usepackage{fancyhdr} \pagestyle{fancyplain} \fancyhead{} \fancyfoot{}

\renewcommand{\headrulewidth}{0pt} % Remove header underlines
\renewcommand{\footrulewidth}{0pt} % Remove footer underlines

\newcommand{\horrule}[1]{\rule{\linewidth}{#1}} % Create horizontal rule command with 1 argument of height

\newcommand{\Pointilles}[1][3]{%
  \multido{}{#1}{\makebox[\linewidth]{\dotfill}\\[\parskip]
}}

\newtheorem{Definition}{Définition}

\usepackage{siunitx}
\sisetup{
    detect-all,
    output-decimal-marker={,},
    group-minimum-digits = 3,
    group-separator={~},
    number-unit-separator={~},
    inter-unit-product={~}
}

\setlength{\columnseprule}{1pt}

\begin{document}

Welcome to Introduction to Applied Cryptography.  Cryptography is an essential component of cybersecurity. The need to protect sensitive information and ensure the integrity of industrial control processes has placed a premium on cybersecurity skills in today’s information technology market.  Demand for cybersecurity jobs is expected to rise 6 million globally by 2019, with a projected shortfall of 1.5 million, according to Symantec, the world’s largest security software vendor. According to Forbes, the cybersecurity market is expected to grow from \$75 billion in 2015 to \$170 billion by 2020. In this specialization, you will learn basic security issues in computer communications, classical cryptographic algorithms, symmetric-key cryptography, public-key cryptography, authentication, and digital signatures. These topics should prove especially useful to you if you are new to cybersecurity Course 1, Classical Cryptosystems, introduces you to basic concepts and terminology related to cryptography and cryptanalysis. It is recommended that you have a basic knowledge of computer science and basic math skills such as algebra and probability.


\section*{Semaine 1 - Introduction}

\subsection*{Introduction to the Course}

This module covers an introduction of the specialization and instructors, covers what to expect from this educational experience and also, an introduction to the course Classical Cryptosystems and Core Concepts.

\begin{itemize}[label={$\bullet$}]
   \item About the Instructors
   \item Course Introduction
   \item About the Course
   \item About the Instructor
   \item About the Instructor
\end{itemize}

\subsection*{Cryptographic Tidbits}

In this module we present an introduction to cryptography, differentiate between codes and ciphers, describe cryptanalysis, and identify the guiding principles of modern cryptography. After completing this course you will be able to read material related to cryptographic systems, understanding the basic terminology and concepts. You will also have an appreciation for the historical framework of modern cryptography and the difficulty of achieving its aims.

\begin{itemize}[label={$\bullet$}]
    \item What is Cryptography?
    \item Lecture Slide - What is Cryptography?
    \item Additional Reference Material
    \item Codes and Ciphers
    \item Lecture Slide - Codes and Ciphers
    \item L2: Additional Reference Material
    \item What is Cryptanalysis?
    \item Lecture Slide - What is Cryptanalysis
    \item L3: Additional Reference Material
    \item Modern Guiding Principles in Cryptography
    \item Lecture Slide - Modern Guiding Principles
    \item L4: Additional Reference Material
    \item Video - Cryptography for the masses: Nadim Kobeissi
    \item Quiz pour s'exercer: Practice Assessment - Cryptographic Tidbits
    \item Discussion Prompt: What do you think?

    \item Noté: Graded Assessment - Cryptographic Tidbits
\end{itemize}

\newpage



\section*{Semaine 2 - cryptanalysis}



Delving deeper into cryptanalysis, in this module we will discuss different types of attacks, explain frequency analysis and different use cases, explain the significance of polyalphabetical ciphers, and discuss the Vigenere Cipher. When you have completed this module, you will have an appreciation of the different types of attacks and under what kinds of situations each might be applicable.

\subsection*{Objectifs d'apprentissage}

\begin{itemize}[label={$\bullet$}]
    \item Identify different types of cryptanalysis attacks.
    \item Describe the single-character frequency analysis method of cryptanalysis.
    \item Explain the multi-character frequency analysis method of cryptanalysis.
    \item Apply the different frequency analysis methods of cryptanalysis.
    \item Explain the significance of key length in polyalphabetic ciphers.
    \item Demonstrate how to determine the key length for a polyalphabetic cipher.
    \item Apply frequency analysis cryptanalysis methods to crack a Vigenere Cipher.
\end{itemize}

\subsection*{Lectures}


\begin{itemize}[label={$\bullet$}]
    \item L5 - Type of attacks
    \item L6 - Frequency Analysis of Monoalphabetic Ciphers - Single-Character Frequencies
    \item L7 - Multi-Character Frequency Analysis
    \item L8 - Frequency Analysis for Monoalphabetic Ciphers - Example
    \item L9 - Key Length Determination in Polyalphabetic Ciphers
    \item L10 - Example of Cracking a Vigenere Cipher
\end{itemize}

\newpage


\end{document}