\documentclass[11pt]{article}
\usepackage{geometry,marginnote} % Pour passer au format A4
\geometry{hmargin=1cm, vmargin=1cm} % 

% Page et encodage
\usepackage[T1]{fontenc} % Use 8-bit encoding that has 256 glyphs
\usepackage[english,french]{babel} % Français et anglais
\usepackage[utf8]{inputenc} 

\usepackage{lmodern,numprint}
\setlength\parindent{0pt}

% Graphiques
\usepackage{graphicx,float,grffile}
\usepackage{pst-eucl, pst-plot,units} 

% Maths et divers
\usepackage{amsmath,amsfonts,amssymb,amsthm,verbatim}
\usepackage{multicol,enumitem,url,eurosym,gensymb}
\DeclareUnicodeCharacter{20AC}{\euro}

% Sections
\usepackage{sectsty} % Allows customizing section commands
\allsectionsfont{\centering \normalfont\scshape}

% Tête et pied de page

\usepackage{fancyhdr} \pagestyle{fancyplain} \fancyhead{} \fancyfoot{}

\renewcommand{\headrulewidth}{0pt} % Remove header underlines
\renewcommand{\footrulewidth}{0pt} % Remove footer underlines

\newcommand{\horrule}[1]{\rule{\linewidth}{#1}} % Create horizontal rule command with 1 argument of height


\begin{document}

\section{ex16}

\textbf{Initialisation :} \\

\subsection*{a)}

Pour $n = 0$, (la famille est vide). $\prod_{i=1}^{0}(1-a_i) = 1$ (produit vide) et $\sum_{i=1}^{0}(a_i) = 0 $ (somme vide). \\

Or $1 \geq 1$ donc la propriété est vérifiée. 

\textbf{Hérédité :} \\
Soit $n \in \mathbb{N}$. Supposons vrai : $\forall (a_i)_{i \in [1,n]} \in [0,1]^{[1,n]}$, on a $\prod_{i=1}^{n}(1-a_i) \geq 1 - \sum_{i=1}^{n}(a_i)$ 


Soit $(a_i)_{i \in [1,n+1]}$ une famille d'éléments de [0,1]. \\

On sait que $\prod_{i=1}^{n}(1-a_i) \geq 1 - \sum_{i=1}^{n}(a_i)$ 

Donc 
\begin{align}
(1 - a_{n+1}) \prod_{i=1}^{n}(1-a_i) &= \prod_{i=1}^{n+1}(1-a_i) \geq (1 - a_{n+1}) (1 - \sum_{i=1}^{n}(a_i)) \\
                                     &= 1 - \sum_{i=1}^{n+1}(a_i) + a_{n+1} \sum_{i=1}^{n}(a_i) \\
                                     &\geq 1 - \sum_{i=1}^{n+1}(a_i) \\
\end{align}
car $a_{n+1} \sum_{i=1}^{n}(a_i) \geq 0$

Donc la propriété est héréditaire et donc vraie $\forall n \in \mathbb{N}$.

\subsection*{b)}

$\sum_{i=1}^{n}(\dfrac{a_i}{m} - 1 ) = \dfrac{1}{m} \sum_{i=1}^{n}(a_i) - \sum_{i=1}^{n}(1) = \dfrac{\sum_{i=1}^{n}(a_i)}{\dfrac{1}{n}\sum_{i=1}^{n}(a_i)} - n = n - n = 0$.

\subsection*{c)}

$\forall i \in [1,n] ln \dfrac{a_i}{n} \leq \dfrac{a_i}{m} - 1$ car $\dfrac{a_i}{m} > 0$.

donc en sommant $\sum_{i=1}^{n}(ln \dfrac{a_i}{m}) \leq \sum_{i=1}^{n}(\dfrac{a_i}{m}- 1) = 0$.

Or, $\sum_{i=1}^{n}(ln \dfrac{a_i}{m}) = ln (\prod_{i=1}^{n}(\dfrac{a_i}{m})) = ln (\dfrac{1}{m^n} \prod_{i=1}^{n}(a_i)) \leq 0 $.

Donc $\dfrac{1}{m^n} \prod_{i=1}^{n}(a_i)) \leq 1 $ donc $\prod_{i=1}^{n}(a_i) \leq m^n$ 

i;e, $ (\prod_{i=1}^{n}(a_i))^{\frac{1}{n}} \leq m$ , CQFD.

\subsection*{d)}

$A(\dfrac{1}{a_1},...,\dfrac{1}{a_n}) = \dfrac{1}{n} \sum_{i=1}^{n}(\dfrac{1}{a_i} = \dfrac{1}{1 / \sum_{i=1}^{n}(\dfrac{1}{a_i})} = \dfrac{1}{H(a_1,..., a_n)}$

Par conséquent, comme $G(\dfrac{1}{a_1},...,\dfrac{1}{a_n}) \leq \dfrac{1}{H(a_1,..., a_n)}$


On a : $H(a_{1},..., a_{n}) \leq \dfrac{1}{ G( \frac{1}{a_1},...,\frac{1}{a_n} ) } = (\prod_{i=1}^{n} \dfrac{1}{a_i})^{-1/n}$ \newline
$= (\prod_{i=1}^{n} (\dfrac{1}{a_i})^{-1})^{1/n}$ \newline
$= (\prod_{i=1}^{n} (a_i))^{1/n}$ \newline
$= G(a_1,...,a_n)$

CQFD



\end{document}