\documentclass[11pt]{article}
\usepackage{geometry} % Pour passer au format A4
\geometry{hmargin=1cm, vmargin=1cm} % 

% Page et encodage
\usepackage[T1]{fontenc} % Use 8-bit encoding that has 256 glyphs
\usepackage[english,french]{babel} % Français et anglais
\usepackage[utf8]{inputenc} 

\usepackage{lmodern}
\setlength\parindent{0pt}

% Maths et divers
\usepackage{amsmath,amsfonts,amssymb,amsthm,verbatim}
\usepackage{multicol,enumitem,url,eurosym,gensymb}

% Sections
\usepackage{sectsty} % Allows customizing section commands
\allsectionsfont{\centering \normalfont\scshape}

% Tête et pied de page

\usepackage{fancyhdr} 
\pagestyle{fancyplain} 

\fancyhead{} % No page header
\fancyfoot{}

\renewcommand{\headrulewidth}{0pt} % Remove header underlines
\renewcommand{\footrulewidth}{0pt} % Remove footer underlines

\newcommand{\horrule}[1]{\rule{\linewidth}{#1}} % Create horizontal rule command with 1 argument of height

%----------------------------------------------------------------------------------------
%   Début du document
%----------------------------------------------------------------------------------------

\begin{document}

\section{Le parcours professionnel}

\subsection{Postes occupés avant l’accès au corps}

\textit{Dans chacun des items ci-­dessus, il précise les éléments de contexte jugés significatifs sur les postes occupés.} \\

\begin{itemize}

  \item Du 15/06/2012 au 15/08/2012 - Stage recherche\\
  \textbf{MmodD - Calcul Numérique} - \textit{Lyon} \\
    Stage de deux mois portant sur des méthodes de décomposition de domaine pour les équations aux dérivées partielles et l'implémentation sous scilab d'une méthode de Dirichlet - Neumann dans le cadre du projet MmodD.

  \item Du 01/02/2012 au 01/07/2012 - Stage appliqué\\
  \textbf{Irstea - Modélisation Hydrologique} - \textit{Lyon} \\
    Stage de cinq mois portant sur "l’étude de l’influence de la variabilité spatiale sur le fonctionnement d’une zone tampon enherbée", par l’investigation d’un modèle hydrologique 3D à base physique et à paramètres distribués :   CATHY-3D.

  \item Du 01/01/2013 au 01/07/2013 - Stage recherche\\
  \textbf{LIRIS - Géométrie Discrète} - \textit{Lyon} \\
    Stage de recherche de six mois portant sur la conception sous Linux et l’implémentation en C++ d’un algorithme de calcul d’alpha-shape en géométrie discrète. 

  \item Du 01/09/2013 au 31/08/2014 - 6 h - Collège\\
    \textbf{Collège Christiane Bernardin} - \textit{Francheville}\\
    admissible - contractuel 2nd degré session exceptionnelle 2014. \\
    En responsabilité d'une classe de $5^{ème}$, découverte de l'AP $6^{ème}$ et utilisation de Labomep.
\end{itemize}


\subsection{Postes occupés pendant le stage - Affectation des stagiaires }

\begin{itemize}
  \item Du 01/09/2014 au 31/08/2015 - 18 h - Lycée\\
    \textbf{Lycée polyvalent Edouard Branly} - \textit{Lyon 5e Arrondissement} \\
\end{itemize}
En responsabilité de 3 classes : $2^{nd}$ GT, $1^{ère}$ STI2D et BTS1 électrotechnique.


\subsection{Postes occupés depuis l’accès au corps}

\begin{itemize}
  \item Du 01/09/2015 au 31/08/2016 - 18 h - Collège
  \begin{itemize}
    \item \textbf{Collège Frédéric Mistral} - \textit{Feyzin - (Zone prévention violence)} - 9h (Principale)
    \item \textbf{Collège Lamartine} - \textit{Villeurbanne - (rep+)} - 9h (Complément de service)
  \end{itemize}
  \item Du 01/09/2016 au 31/08/2019 - 18 h - Collège\\
    \textbf{Collège Frédéric Mistral} - \textit{Feyzin - (Zone prévention violence)} - 18h
  \item Depuis 01/09/2019 - 18 h - Collège\\
    \textbf{Collège Faubert} - \textit{Villefranche-sur-Saône - (Réseau d'éducation prioritaire)} - 18h \\
\end{itemize}
En responsabilité de classe de tous les niveaux avec une petite préférence pour les "grands" : $3^{ème}$ et $4^{ème}$. Professeur principal de 2015 à 2020. Référent numérique et réseau de 2015 à 2019 et membre projet DFIE Compétences au collège Frédéric Mistral à Feyzin.

\newpage

\section{Compétences mises en œuvre dans le cadre de son parcours professionnel}

\subsection{L’agent dans son environnement professionnel propre}

\textsc{la classe, le CDI, la vie scolaire, le CIO) : compétences liées à la maitrise des enseignements, compétences scientifiques, didactiques, pédagogiques, éducatives et techniques.}\\

\textit{L’agent expose les réalisations et les démarches qui lui paraissent déterminantes pour caractériser la mise en œuvre de ses compétences et leur contribution aux progrès et au développement de tous les élèves (20 lignes maximum)}\\

J'ai pour attachement d'essayer deux choses : Laisser les élèves le plus longtemps possible en exercice, en autonomie. Et avoir un temps d'écoute, de correction, de cours plus court mais avec beaucoup d'attention.


\subsubsection{Pour les troisièmes}

Une problématique : Des chapitres qui s'enchaînent et ont tendance à se remplacer et non s'empiler. Un fait : le DNB. 

J'ai choisi cette année de faire de petits chapitres, des évaluations régulières de connaissances. En parallèle, je fais chercher beaucoup de problèmes de type brevet, pas nécessairement en lien direct avec le cours lors des séances en demi-groupe. 

Utilisation intensive de la calculatrice, des fonctions avancées, des limites. Insister sur la rédaction des calculs. 

\subsubsection{Pour les cinquièmes}

Je commence le cours par une phrase du jour. Un concept importé des collègues de français. Une phrase : definition, rappel, anecdote, citation sur le thème des maths à chaque début de cours en 5è.

Utilisation régulière de la calculatrice et faire la distinction entre le résultat et les méthodes, la rédaction d'un calcul.

Faire prendre conscience de l'importance de revenir régulièrement sur son cahier et pas seulement pour les évals ni même que pendant le cours avec l'utilisation d'un quadrillage à griser lors de la relecture. 

Beaucoup de travaux graphiques de représentation. 

\subsubsection{Points à améliorer}

La tenue des cahiers.

\newpage

\subsection{L’agent inscrit dans une dimension collective}

\textit{L’agent s’appuie sur quelques exemples concrets et contextualisés pour analyser sa participation au suivi des élèves, à la vie de l’école/l’établissement et son implication dans les relations avec les partenaires et l’environnement (20 lignes maximum).}\\

Depuis ma première affectation à Feyzin en 2015, j’ai été \textbf{professeur principal} selon les besoins. J'ai particulièrement apprécié travailler sur le projet d’orientation des élèves de $3^{ème}$. Le rapport à la famille, s'entretenir auprès des collègues pour savoir comme ça se passe dans leur cours.\\

La découverte, la participation puis la labélisation en 2019 par la DFIE du \textbf{projet compétence} a rythmé mon travail au collège Frédéric Mistral à Feyzin. La première année, j'ai principalement assisté aux réunions la première année pour comprendre la notation, la technique avec pronote. Je me suis intéressé à la notion de compétences. Je me suis alors impliqué dans le projet DFIE, son évaluation et sa labélisation la dernière année qui nous demandé un travail conséquent.\\

Je me suis investissement dès septembre la première année comme \textbf{référent numérique et réseau} sur le collège de Feyzin sur à la mutation de ce dernier. J'ai suivi la formation DANE la première année. J'ai été un interlocuteur du gestionnaire, des collègues sur les questions techniques et j'ai participer à la mise en place de différents tests : pisa, éval 6è,...\\

\subsubsection{Le rôle de professeur principal}

\begin{itemize}
  \item 2015 - 2016 : 4è - Frédéric Mistral à Feyzin
  \item 2016 - 2017 : 5è - Frédéric Mistral à Feyzin
  \item 2017 - 2018 : 3è - Frédéric Mistral à Feyzin
  \item 2018 - 2019 : 3è - Frédéric Mistral à Feyzin
  \item 2019 - 2020 : 4è - Faubert à Villefranche-sur-Saône
\end{itemize}

\subsubsection{Référent numérique et réseau}

\begin{itemize}
  \item 2015 - 2019 : Frédéric Mistral à Feyzin.
\end{itemize}

\subsubsection{Cardie / DFIE}

\begin{itemize}
  \item 2015 - 2019 : Projet DFIE Compétences
    Participation à la labélisation du projet en 2019
  \item 2017 - 2019 : Chef de projet DFIE : Périscolaire.
\end{itemize}

\subsubsection{Autres}

\begin{itemize}
  \item 2015 - 2019 : Membre élu au CA dans diverses commissions : commission permanente, le conseil de discipline, le conseil pédagogique, appel d'offre,...
  \item 2016 - 2019 : Atelier Jeux de société : Périscolaire.
  \item 2016 - 2019 : Acteur dans le projet médiation par les pairs sur le niveau 5è.
  \item 2016 - 2019 : Participation à la liaison école/collège.
  \item 2020 - 2021 : Coordinateur de matière - Faubert à Villefranche-sur-Saône
\end{itemize}

\newpage

\subsection{L’agent et son engagement dans une démarche individuelle et collective de développement professionnel}

\textit{L’agent décrit les démarches accomplies pour développer cette compétence telle qu’explicitée dans le référentiel et formule ses besoins d’accompagnement (10 lignes maximum).}\\

Chaque année, je profite du PAF pour suivre des formations. Cette année avec une certaine appréhension du covid, je ne me suis pas inscrit. 

\subsubsection{Astronomie}

J'ai grandement apprécié les formations sur le thème de l'astronomie proposées par le CRAL. Je continue à me former sur le sujet via la plateforme coursera. J'aimerai sensibiliser les élèves à ce domaine. 

\begin{itemize}
  \item Faire de l'astronomie en AP, dans les EPI, MPS ou TPE - CRAL - 2016
  \item Ce que nous apprend la recherche des exoplanets - CRAL - 2015
  \item Astronomie : explorer le temps et l'espace et Archaeoastronomy - Coursera (mooc) - 2019
\end{itemize}


\subsubsection{Un besoin dans l'établissement}

J'ai suivi des formations en lien direct avec mon travail au sein de l'établissement. 

\begin{itemize}
  \item Formation professeurs principaux de troisième et seconde pro - DAFOP - 2019
  \item Colloque Académique sur le numérique éducatif - DANE - 2016
  \item Gestion du réseau scribe - DANE - 2015
  \item Regroupement des équipe en expérimentation  - DFIE - 2017
  \item Évènements de l'innovation - DFIE - 2019
\end{itemize}

\subsubsection{En mathématiques}

J'ai la première année suivi une très bonne formation proposée par l'IREM sur le calcul mental. Depuis, je continue chaque année à m'entretenir en maths. Je participe activement à un discord centré sur les  mathématiques où je continue à être en contact avec des élèves de lycée, du supérieur et aussi un professeur de lycée qui l'administre. 

\begin{itemize}
  \item Du calcul mental à la mise - IREM - 2014
  \item Mathématiques : préparation à l'entrée dans l'enseignement supérieur - FUN (mooc) - 2017
\end{itemize}

\subsubsection{En transversal}

J'ai suivi une très bonne formation sur la gestion de la voix l'année dernière. Je pense peut-être continuez sur une formation sur le théâtre. 

\begin{itemize}
  \item Connaître, entretenir et améliorer sa voix - 2020
\end{itemize}

\newpage

\section{Souhait(s) d’évolution professionnelle, de diversification des fonctions}

\textit{L’agent qui le souhaite formule ses souhaits d’évolution professionnelle et de diversification des fonctions : tuteur, coordonnateur, formateur, formateur académique, mobilité vers d’autres types d’établissement scolaires, vers d’autres publics (établissement en EP, élèves à besoins éducatifs particuliers, collège, lycée, post bac, enseignement à l’étranger,...), vers d’autres métiers de l’enseignement, vers les corps d’encadrement, vers d’autres corps de la fonction publique, etc. (20 lignes maximum).}\\

 Je suis assez partagé sur la question. \\

 Je me sens à ma place là où je suis. Je ressens mon utilité. J'apprécie le challenge que propose l'enseignement des mathématiques au collège notamment dans un établissement d'un réseau prioritaire. J'ai une légère préférence pour les "grands" avec l'enseignement des grands théorèmes, la notation scientifique, les statistiques ainsi que la préparation du brevet ainsi que de la seconde. \\

 D'un autre côté, j'ai encore cette expérience du lycée qui a été un plaisir pour moi. Suite à la réforme, l'ajout de SNT, de Python sont des parties qui m'attirent également. Je me tiens très à jour des sorties et des nouveautés liés aux calculatrices graphiques. Je suis en contact régulier avec des élèves de lycées pour proposer un peu d'aide en ligne sur un discord de maths. Un enseignant de maths de lycée est également présent. 

\end{document}